Graph-based model transformations have become in the last few years the main means
for manipulating models in model-driven development. Their simplicity,
their allowance for mathematical treatment, and the fact that they can natively
manipulate domain-specific concepts expressed in metamodels, all make graph-based
model transformations an excellent compromise between strong theoretical
foundations and applicability to real-world problems.

Because of the importance of model transformations both in the academic and the industrial arenas,
their verification is of prime importance: firstly because
the correctness of software built using model-driven development techniques
typically relies on the correctness of many operations executed using model
transformations; and secondly because tools that allow building verified
software are in strong demand, especially in industry where quality and security
standards have to be met.

The tool we present can verifying these transformations with regard to pre- / post- condition patterns. A contract holds in the transformation if, for all input models where the contract's pre-condition
is found, the contract's post-condition is also found in the
corresponding output model (with optional traceability constraints between the
elements of the input and output models). Otherwise, the contract does not hold.

%For example, this paper considers the well-known \emph{Families-to-Persons}
%transformation from the ATL zoo~\cite{atlZoo}, where mother(s), father(s),
%daughter(s) and son(s) belonging to a family are translated into men and women
%who are members of a community. One possible contract would try to assert that,
%for any input model containing a family that includes a mother and a daughter,
%a man is produced in the output community. We would expect the contract not to
%hold for the \emph{Families-to-Persons} transformation, because there exist families that are composed of
%only a mother and her daughter.
%
%The main contribution of the technique we present here is that, if
%our prover demonstrates that the contract holds, then it will hold for any input
%model given to the ATL model transformation. We can thus guarantee the
%user can safely execute the model transformation without any need for additional
%testing or runtime checking. Our contract language is based on pre- /
%post-condition contracts, but also includes propositional logic operators
%for combining contracts as we explain further ahead in this paper.

We prove that contracts hold or not for transformations constructed in a model transformation language called
DSLTrans~\cite{Barroca2011}.  A theoretical framework has been developed for
the DSLTrans model transformation language in which pre- / post- condition
contracts can be shown to hold for all input/output pairs resulting from executing a given
DSLTrans model transformation, or to not hold for at least one of those
input/output pairs~\cite{Lucio2014}. A fully automatic property prover based on
this theory has been shown to be applicable to industrial
problems~\cite{Selim2014}.


