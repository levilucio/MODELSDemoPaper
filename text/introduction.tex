% Graph-based model transformations have become in the last few years the main
% means for manipulating models in model-driven development. Their simplicity,
% their allowance for mathematical treatment, and the fact that they can natively
% manipulate domain-specific concepts expressed in metamodels, all make
% graph-based model transformations an excellent compromise between strong
% theoretical foundations and applicability to real-world problems.
% Because of the importance of graph-based model transformations both in the
% academic and the industrial arenas, their verification is of prime importance.
In this paper we introduce SyVOLT (Symbolic Verifier of mOdeL Transformations),
an Eclipse plugin that allows verifying pre-/post- condition structural
contracts on model transformations, specified in the DSLTrans transformation
language. SyVOLT's operation relies on a theoretical framework has been
developed for the DSLTrans model transformation language. In this framework,
pre- / post- condition contracts can be shown to either hold for all input/output pairs resulting from
executing a given DSLTrans model transformation, or not to hold for at least
one of those input/output pairs~\cite{Lucio2014}.

% A fully
% automatic property prover based on this theory has been shown to be applicable
% to industrial problems~\cite{Selim2014}.

There has been already an extensive work on verifying different aspects of model
transformations, e.g., cf.~\cite{AmraniLSCDVTC12} for a survey.
With respect to the contribution of this paper, we summarize previous
contributions for checking different kinds of contracts for model
transformations, where the concrete approaches range from testing to
verification approaches.

In~\cite{Gogolla2011,Vallecillo2012} the authors describe a method where
`Tracts' can be specified for model transformations. Tracts define a set of
constraints on the source and target metamodels, a set of source-target
constraints, and a tract test suite, i.e., a collection of source models
satisfying the source constraints. The accompanying TractsTool can then
automatically transform the source models into the target metamodel, and
subsequently verify that the source/target model pairs satisfy the constraints.
The advantages of this are that the approach is not computationally-intensive,
as tests can be narrowly focused in a modular way. Besides the Tracts approach,
there are several other approaches supporting the testing of model
transformations based on different kind of contracts such as model
fragments~\cite{Mottu2008}, graph patterns~\cite{Guerra12,BaloghBCGHMPPRVa10},
Triple Graph Grammars (TGGs)~\cite{WieberAS14}, dedicated testing
languages~\cite{Kolovos06,Garcia-Dominguez11}, or as used in Tracts OCL
constraints~\cite{Cariou09}, and even a combination of these mentioned
approaches~\cite{Pele}. While these mentioned approaches resort to black-box
based testing, there are also approaches which allow for white-box based testing
of model transformations such as~\cite{GonzalezC12}.

In contrast to testing approaches, SyVOLT allows for contracts to
be proved for all possible transformation executions, i.e., for all possible
input models. However, we also keep same implication idea: the pre-condition of
a contract sets constraints on the input models to the transformation, and then,
the post-condition defines constraints on the output model.


