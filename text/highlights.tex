SyVOLT has a number of unique features, outlined below.

\subsection{Graphical Modelling of Model Transformation Contracts}

SyVOLT proves pre-/post-condition contracts hold for a model transformation.
Such contracts establish relations between patterns occurring in input and output
models of a model transformation. If a contract holds, a formal guarantee
exists that whenever a transformation's input model contains the pattern
specified in the pre-condition of the contract, its the output model contains
the pattern specified in the contract's post-condition. Contracts can 
optionally include traceability relations between input and output patterns. The
visual representation of a contract in the SyVOLT editor is meant for allowing
the user to intuitively understand its meaning.
%If a textual (logical or mathematical) editor where to be used, the user would
%need an extra system of identifiers to correctly prescribe the associations
% between pre and post-condition elements whereas in the visual representation, the user graphically builds the associations between those elements.
%\levi{Claudio, I don't understand this sentence}
%\cgg{Please tell me now if you can understand it.}

%The visual representation of a contract has all the necessary information to
% derive the correct logical expression to be used by the internal SyVolt prover.

\subsection{Push-Button Proofs}
\label{sec:push_button_proofs}
The proving process for a SyVOLT contract is fully automatic and the all of the
approach's formal details are completely hidden from the user. Once the
transformation and the contracts of interest are created, one command will start
the property proving process. This process will automatically create all
required artifacts (as detailed in Section~\ref{sec:arch}), run the process, and
provide the results to the user within the Eclipse environment. This allows the user to continually stay within the
Eclipse environment, where he or she develops the contracts and the
model transformations.




\subsection{Based on Symbolic Execution}

Our technique shares its principles with symbolic execution, a classic method to
verify code. The underlying idea entails building a finite representation of the
(infinite) set of computations that can be built by a model transformation
specification. In this context, each symbolic execution -- which in the context
of our work we call a \emph{path condition} -- consists of a combination
of a subset of the transformation's rules.
Because a path condition contains a number of rules, it represents the execution
of the model transformation over any input model all those rules match on.
Properties of interest are proved on the set of path conditions built for a
model transformation, and are extrapolated to the infinite set of
the model transformation's computations through an abstration relation.

% Our model transformation verification technique relies on models as the
% means to internally represent both symbolic executions and contracts. SyVOLT
% then reasons over these graphs to build a proof that contracts hold or do not
% hold.

% Note that by using a typed graph representation, our technique can prove
% contracts that include constraints on the structure and the attributes of the
% input and output models. 

% \levi{keep the text from here on?} An example of this would be in the
% Families-To-Persons transformation from the ATL zoo [CITE]. In this transformation, the name for a person in the
% output graph is a concatenation of two strings \cgg{instead of ''strings``,
% should be ''attributes``} from elements in the input graph. Our contract prover
% can prove that this concatenation will be valid in all cases.



\subsection{Input Independence and Exhaustiveness} 

Our technique is exhaustive, in the sense that whenever a contract holds, it
will hold for all possible input models to a transformation. This is possible
because SyVOLT operates on specifications of outplace model transformations,
where unbounded loops and model element deletions are not allowed. SyVOLT proves
contracts in an input-independent manner, relying only on the specification of the
transformation. The approach is sound and complete, as described
in~\cite{Lucio2014}.  Figure~\ref{fig:output} shows an example of the output presented to the user, detailing which contracts hold and do not hold.

\begin{figure}
%\centering
%\includegraphics[width=0.5\textwidth]{figures/output}

\scriptsize
\tt
Proving contracts:

Contract ``daughterMother'' holds for all input models!
\\
Contract ``motherFather'' does NOT hold for all input models! The contract fails on the following Path Conditions:
[`EmptyPathCondition\_RootRule\_FatherRule\_MotherRule', ...]

The smallest Path Condition where the contract fails are:
[`EmptyPathCondition\_FatherRule\_MotherRule']
\\
Time to verify 2 contracts: 11.6834638966 seconds.
It took 0.769201040268 seconds to build the set of path conditions.
\caption{Sample results of the contract prover}
\label{fig:output}
\end{figure}

\subsection{Proving Contracts about ATL Model Transformations}
The Atlas Transformation Language (ATL\footnote{http://eclipse.org/atl}) is
commonly-used in both industry and academia applications. In order to extend our approach into these domains, we
have developed a higher-order transformation that is able to automatically
transform declarative ATL transformations\footnote{See ~\cite{Oakes} for the
constructs of ATL that SyVOLT can verify.} into our transformation language
DSLTrans. This allows the user to also prove contracts on ATL transformations. Extending this capability to SyVOLT's user interface is
future work.

\subsection{Scalability and Speed}

We have some evidence that SyVOLT scales to transformations of practical
interest, more scientific work to fully validate this hypothesis is
necessary. In particular we have applied DSLTrans transformations with up to
over 60 rules, and ATL transformations with up to 13 rules~\cite{Oakes}. From our own
experience with DSLTrans, the size of a DSLTrans transformations varies widely,
with the average size ranging from 10 up to 50 rules. The average size of an ATL
transformation is around 20 rules. [cite Manuel's paper??].
Even though our technique is exhaustive, our approach takes relatively short
amounts of time to prove contracts. For example, our experiments with
transformation for the ATL
Zoo\footnote{http://www.eclipse.org/atl/atlTransformations/}~\cite{Oakes} show that contracts can be verified within seconds. In Gehan Selim's PhD thesis~\cite{Selim2015}
further evidence of SyVOLT's speed is given when verifying a relatively large model transformation
for giving semantics to the UML-RT language in terms of the Kiltera process
language~\cite{PosseDingel2014}. SyVOLT's symbolic execution engine is fully
homegrown~\cite{LucioVang} and does not depend on third-party solvers. Although
this has implied a large effort to build the codebase, it has allowed us to
have the required control over the code to iteratively optimize the engine for
space and time economy.
\cite{Selim2014} demonstrates that our prover is substantially faster than
similar approaches based on SAT solvers.


\subsection{Counter-Examples}

When a given contract is proved to be violated by a given model transformation,
SyVOLT can produces additional information for the user to pinpoint where
the contract's violation occurs. This information is in
the form of the set of model transformation rules used to build a particular
path condition for which the contract fails. A counter-example is any input model that can be
consumed all the rules in this set.

%  This means that any input model that belongs to the family of the
% counter-example - i.e., fits the description given by the contract proving
% process - causes the model transformation to violate the contract.
% We suggest that this supports a transformation
% development method analogous to 'test-driven development'. In this method,
% development would be routinely punctuated by contract proof in order to catch
% errors early and store test cases - the counter examples produced - to be used
% in the future.


\subsection{Integration with Eclipse}

Eclipse is a popular development environment and model transformation
tools such as ATL, DSLTrans~\cite{Barroca2011} and
EGL~\footnote{http://www.eclipse.org/epsilon/} are integrated with the Eclipse Modeling Framework
(EMF)~\footnote{https://eclipse.org/modeling/emf/}. To take advantage of this
ecosystem, SyVOLT integrates with EMF too. In the EMF, models can be represented in a multitude of syntaxes, from
graphical to textual, and this makes the interaction with SyVOLT easier since the modeler
can use the model editor that he/she finds most convenient.
%Internally, SyVOLT uses the Himesis format~\cite{Provost2006} to represent
% models.

%\subsection{Model Driven Developed GUI\levi{this text is subsumed by section
%III a)}}
%\label{sec:mdd_gui}

%To take advantage of the productivity promised by MDD, we used a language
% called Eugenia[CITE] to develop the SyVolt contract editor shown in
%Figure~\ref{fig:eclipse_frontend}.
%With this approach, the SyVolt editor was developed in about 4 man-hours.

\begin{figure}
\centering
\includegraphics[width=0.5\textwidth]{figures/eclipse_frontend}
\caption{The SyVOLT editor within Eclipse}
\label{fig:eclipse_frontend}
\end{figure}




 