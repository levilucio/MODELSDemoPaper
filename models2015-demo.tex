\documentclass[conference]{IEEEtran}
%\usepackage[latin1]{inputenc}
\usepackage{amsmath}
\usepackage{amsfonts}
\usepackage{amssymb}
\usepackage{graphicx}
\usepackage{epstopdf}
\usepackage{AMMALanguages}
\usepackage[table]{xcolor}
\usepackage{tabularx}
\usepackage{booktabs}
\usepackage{cleveref}
\usepackage{url}

\input{commentSupport}

\newcommand{\chunk}[2]{%
	\fcolorbox{black}{yellow}{\bfseries\sffamily\scriptsize#1}%
   {$\blacktriangleright$#2$\blacktriangleleft$}%
}

\newcommand{\bentley}[1]{\chunk{Bentley}{\textbf{\textcolor{green}{\textsl{#1}}}}}
\newcommand{\levi}[1]{\chunk{Levi}{\textbf{\textcolor{red}{\textsl{#1}}}}}
\newcommand{\manuel}[1]{\chunk{Manuel}{\textbf{\textcolor{blue}{\textsl{#1}}}}}
\newcommand{\javi}[1]{\chunk{Javi}{\textbf{\textcolor{brown}{\textsl{#1}}}}}

\begin{document}
\title{SyVOLT - A Model Transformation Verifier}
%\title{Full Verification of Model Transformation Contracts for Declarative ATL}
\author{Levi L\'{u}cio, Bentley James Oakes, Claudio Gomes, Hans Vangheluwe}

\author{\IEEEauthorblockN{Levi L\'{u}cio\IEEEauthorrefmark{1},
			Bentley James Oakes\IEEEauthorrefmark{1}, and
			Claudio Gomes\IEEEauthorrefmark{2}}
		\IEEEauthorblockA{\IEEEauthorrefmark{1}School of Computer Science,
			McGill University,
			Canada\\ levi@cs.mcgill.ca, bentley.oakes@mail.mcgill.ca}
		\IEEEauthorblockA{\IEEEauthorrefmark{2}MSDL Lab,
			U of Antwerp,
			Antwerp\\ \{claudio\}@internet}
	}

\date{\today}


\maketitle

%\tableofcontents



\begin{abstract}

%The Atlas Transformation Language (ATL) is today a de-facto standard in
%model-driven development. It is understood by the community that methods for
%exhaustively verifying such transformations provide an important pillar for
%achieving a stronger adoption of model-driven development in industry.
%
%In this paper we propose a method for verifying ATL model transformations by
%translating them into DSLTrans, a transformation language with limited
%expressiveness. Pre- / post- condition contracts are then verified on the
%resulting DSLTrans specification using a symbolic-execution property prover.
%
%The technique we present in this paper is exhaustive for the declarative ATL
%subset, meaning we are certain that if a contract holds, it will hold when any
%input model is passed to the ATL transformation being checked. We explore the
%scalability of our technique using a set of examples, including a model
%transformation developed in collaboration with our industrial partner.



We introduce SyVOLT, an Eclipse plugin for the verification of structural contracts on model transformations. The plugin allows the user to build transformations within the Eclipse environment, in our transformation language DSLTrans. The pre-/post- condition contracts can also be built in a similar interface. The contract prover is exhaustive, meaning that if a contract holds, then the contract will hold for all input models of a transformation. The contract prover can also present in which input model counter-examples the property will fail.


Demo: \url{https://www.youtube.com/watch?v=8PrR5RhPptY}



% The Atlas Transformation Language (ATL) is today one of the most-used model
% transformation languages and has become a de-facto standard in model-driven
% development. It is understood by the community that methods for exhaustively
% verifying such transformations provide an important pillar for achieving a
% stronger adoption of model-driven development in industry.
% Proposals for the verification of ATL transformations have arisen in the past
% few years. However, these techniques are either based on non-exhaustive testing
% or on proof methods that require human assistance and/or are not complete.
%
% In this paper we propose a method for exhaustively verifying ATL model
% transformations by translating them into DSLTrans, a transformation language
% with limited expressiveness. The properties of interest are then verified on the
% resulting DSLTrans specification by using a symbolic-execution property prover.
%
% The technique we present in this paper is complete for the declarative ATL
% subset, meaning we are certain that a pre- post-condition contract either holds
% or does not hold for all input models of a given ATL transformation. Note that
% as the metamodel language is the same for both transformations, these contracts
% are applicable seamlessly in both the ATL and DSLTrans contexts. As well, we
% explore the scalability of our technique by presenting experimental evidence of
% applying the technique to a model transformation developed in collaboration with
% our industrial partner.
\end{abstract}


\section{Introduction}
\label{sec:intro}
% Graph-based model transformations have become in the last few years the main
% means for manipulating models in model-driven development. Their simplicity,
% their allowance for mathematical treatment, and the fact that they can natively
% manipulate domain-specific concepts expressed in metamodels, all make
% graph-based model transformations an excellent compromise between strong
% theoretical foundations and applicability to real-world problems.
% Because of the importance of graph-based model transformations both in the
% academic and the industrial arenas, their verification is of prime importance.
In this paper we introduce SyVOLT (Symbolic Verifier of mOdeL Transformations),
a user-friendly Eclipse plugin to verify pre-/post- condition structural contracts
on model transformations specified in the DSLTrans transformation language.
SyVOLT's operation relies on a theoretical framework has been developed for the
DSLTrans model transformation language, in which pre- / post- condition
contracts can be shown to hold for all input/output pairs resulting from
executing a given DSLTrans model transformation, or to not hold for at least one
of those input/output pairs~\cite{Lucio2014}.

% A fully
% automatic property prover based on this theory has been shown to be applicable
% to industrial problems~\cite{Selim2014}.

There has been already an extensive work on verifying different aspects of model
transformations, e.g., cf.~\cite{AmraniLSCDVTC12} for a survey.
With respect to the contribution of this paper, we summarize previous
contributions for checking different kind of contracts for model transformations
whereas the concrete approaches range from testing to verification approaches.

In~\cite{Gogolla2011,Vallecillo2012} the authors describe a method where
`Tracts' can be specified for model transformations. Tracts define a set of
constraints on the source and target metamodels, a set of source-target
constraints, and a tract test suite, i.e., a collection of source models
satisfying the source constraints. The accompanying TractsTool can then
automatically transform the source models into the target metamodel, and
subsequently verify that the source/target model pairs satisfy the constraints.
The advantages of this are that the approach is not computationally-intensive,
as tests can be narrowly focused in a modular way. Besides the Tracts approach,
there are several other approaches supporting the testing of model
transformations based on different kind of contracts such as model
fragments~\cite{Mottu2008}, graph patterns~\cite{Guerra12,BaloghBCGHMPPRVa10},
Triple Graph Grammars (TGGs)~\cite{WieberAS14}, dedicated testing
languages~\cite{Kolovos06,Garcia-Dominguez11}, or as used in Tracts OCL
constraints~\cite{Cariou09}, and even a combination of these mentioned
approaches~\cite{Pele}. While these mentioned approaches resort to black-box
based testing, there are also approaches which allow for white-box based testing
of model transformations such as~\cite{GonzalezC12}.

In contrast to testing approaches, SyVOLT allows for contracts to
be proved for all possible transformation executions, i.e., for all possible
input models. However, we also keep same implication idea: the pre-condition of
a contract sets constraints on the input models to the transformation, and then,
the post-condition defines constraints on the output model.




\section{Highlights}
\label{sec:highlights}
SyVOLT is a user-friendly Eclipse plugin to verify pre-/post- condition
contracts on model transformations specified in the DSLTrans transformation
language. The tool has a number of unique features, outlined below.

\subsection{Integration with Eclipse}

Eclipse is a popular development environment and model transformation
tools such as ATL [CITE], DSLTrans [CITE] and EGL [CITE] are integrated with the
Eclipse Modeling Framework (EMF) [CITE].
To take advantage of this ecosystem, SyVolt also integrates with EMF.

In the EMF, models can be represented in a multitude of syntaxes, from graphical
to textual, and this makes the interaction with SyVolt easier since the modeler
can use the model editor that is most convenient. Internally, SyVolt uses 
the Himesis format [CITE] to represent models.

\subsection{Eclipse Frontend}

\begin{figure}
\centering
\includegraphics[width=0.45\textwidth]{figures/eclipse_frontend}
\caption{The transformation editor within Eclipse}
\label{fig:eclipse_frontend}
\end{figure}

Figure~\ref{fig:eclipse_frontend} shows the Eclipse frontend, where a user can
edit a DSLTrans transformation within the Eclipse editor. A similar view is also
used for users to define the contracts for their transformations. Note that
these editors are based within the model creation framework FIX FIX.

\subsection{Graphical Modelling}
Both the transformation and the contracts are built using a graphical
environment. This intuitive representation allows the user to quickly and easily
visualize and construct the required contracts. This is in contrast to  the
logical or mathematical expression approach which is required by other
verification techniques.

\subsection{Import from ATL}
The Atlas Transformation Language (ATL) is commonly-used in both industry and
academia applications. In order to extend our approach into these domains, we
have developed a higher-order transformation that is able to automatically
transform declarative ATL transformations into our transformation language
DSLTrans~\cite{Oakes}. This allows the user to exhaustively prove contracts on
ATL transformations.

\subsection{Handling of String Attributes}
The DSLTrans transformations and contracts are built upon typed graphs.
Our contract prover is able to reason over these graphs to prove contracts. An
example of this would be in the Families-To-Persons transformation from the ATL
zoo [CITE]. In this transformation, the name for a person in the output graph is
a concatenation of two strings from elements in the input graph. Our contract
prover can prove that this concatenation will be valid in all cases.

\subsection{Push-Button Proving}
Once the transformation and the contracts of interest are created, one command
will start the property proving process. This process will automatically create
all required artifacts (as detailed in the following section), run the process,
and then provide the results to the user within the Eclipse environment, as seen
in Figure~\ref{fig:output}. This allows the user to continually stay within the
Eclipse environment.

\begin{figure}
\centering
\includegraphics[width=0.45\textwidth]{figures/output}
\caption{The results of the contract prover}
\label{fig:output}
\end{figure}

\subsection{Input Independence and Exhaustiveness}

Our technique will exhaustively prove whether a contract will hold for all
possible input models to a transformation. This allows us to exhaustively verify
a transformation, as seen in~\cite{Lucio2014}.

\subsection{Proving Speed}

 Even though our technique is exhaustive, our approach takes a relatively short
 amount of time to prove contracts. For example, our experiments on industrial
 transformations~\cite{Oakes} show that contracts can be verified within a few
 minutes. \cite{Selim2014} demonstrates that our prover is faster than similar
 mathematical approaches.

\subsection{Production of Counter-Examples}

An advantage of our technique is that a counter-example model for a particular
contract can be produced by the contract proving process. This allows the user
to easily determine the error in the transformation and correct it. We suggest
that this enables a transformation development method analogous to 'test-driven
development'. In this method, development would be routinely punctuated by
contract proof in order to catch errors early.

 

\section{Architecture}
\label{sec:arch}


The guiding principle of the McGill Unversity Modelling, Design, and Simulation
Lab is that all tools and processes should be explicitly modelled at an
appropriate abstraction level. This practice ensures that software development
effort is targeted at the problem's essential complexity, while alleviating complexity induced by the computational platforms used.

%This practice goes in the direction of ensuring that essential
%complexity is the main object of software development and that accidental
%complexity is mitigated.

SyVOLT's codebase has been developed by applying these principles. We have used
available model-driven development technology as much as possible, both to
develop and as a part of SyVOLT itself. In particular, we have made it such that
the DSLTrans transformations, the SyVOLT contracts, and path conditions are all represented and treated as models by the proof engine.
 Moreover, we have operationally encoded the
 operations required by the algorithm for building a contract proof (described
 in~\cite{Lucio2014}) as model transformations~\cite{LucioVang}.

The following model-driven development tools have been used in SyVOLT's
development:
\begin{itemize}

%   \item \textbf{AToM$^3$}~\cite{atom3:2002}: AToM$^3$ is a meta-editor for
%   model-driven development.
% %   It has been developed at the MSDL and is used for constructing,
% %   models, metamodels and model transformation rules, and for automatically
% %   synthesising modeling environments.
%   AToM$^3$ has been extensively used to give support to the construction and
%   visualisation of models and model transformation rules required by the proof
%   engine during the initial stages of the construction of SyVOLT. 
% %   Many of these artifacts were explicitely built during the construction
% %   of SyVOLT in order to develop and test the proof algorithm in a controlled
% %   environment. In the finished SyVOLT all these artifacts are automatically
% %   generated from their graphical representations in the Eclipse front-end.\\

  \item \textbf{Himesis}~\cite{Provost2006}: Himesis is a typed graph representation
  format, built upon the open-source igraph\footnote{http://igraph.org/} library. 
% Himesis is used for several reasons: independence, efficiency and tool support.
% Independence is ensured because Himesis is very expressive and also very simple,
% which means it will not change for the foreseeable future.
In \cite{Syriani2010b}, it is reported empirically that Himesis is a good format
to perform the typical graph manipulation operations.
% and it is seamlessly supported by T-Core.
  Himesis is used pervasively within SyVOLT to represent all models 
  and model transformation rules required by the proof algorithm.
  
  \item \textbf{T-Core}~\cite{Syriani2010a}: T-Core is a collection of model transformation
  primitives allowing fine-grained manipulation of models represented in the
  Himesis format.
%   It was initially built as a means to rapidly build
%   high-level model transformation languages. T-Core has been used to
%   successfully deconstruct and reconstruct several mainstream model transformation
%   languages~\cite{}. 
  The main operations of T-Core are model \emph{matching},
  model \emph{rewriting} and \emph{iterating} through a set of match sites in a model.
  The level of control in model manipulation, together with T-Core's speed and
  scalability when treating large models, suited our needs well when
  implementing the property proof algorithm described in~\cite{Lucio2014}. Note that because
  T-Core is also explicitly modelled, a T-Core model transformation rule is also
  a (Himesis) model.

  \item \textbf{Eclipse Modelling Framework} (EMF): SyVOLT makes
  use of EMF's E-Core format for the XMI representation of DSLTrans transformations
  and SyVOLT's contracts within the Eclipse editors.
%   Note that EMF is only used
%   in the Eclipse front-end, and that Ecore models are converted into
%   the Himesis format for the contract prover engine to compute the
%   proofs.\\
  \item \textbf{Epsilon Generation Language} (EGL): Converting
  Ecore models into Himesis models is achieved using EGL, a
  model-to-text transformation language.
% Besides easy integration with Eclipse and native support of the expressive
% Epsilon Object Language (EOL~\cite{Kolovos}), EGL also provides tasks for the
% Ant~\footnote{\url{http://ant.apache.org}} build tool, which was used to
% orchestrate the multiple tools in the proving process (see
% Section~\ref{sec:push_button_proofs}).\\
\end{itemize}

\begin{figure}
\centering
\includegraphics[width=0.5\textwidth]{figures/tooling_arch}
\caption{The architecture of the SyVOLT tool}
\label{fig:arch}
\vspace{-.5cm}
\end{figure}


% Elements of the architecture that could not be built using model driven tools
% (for lack of technological availability) have been developed using model-driven
% principles.

Figure~\ref{fig:arch} shows the architecture of our tool, where
squares containing gears represent computations and squares
containing no gears represent produced and consumed data.
Two essential blocks are depicted: the graphical Eclipse front-end for user
interaction and the proof engine back-end which implements the proving algorithm. The front-end and the back-end communicate in the following way:
in the left-to-right direction, a number of artifacts (models and model
transformations) are synthesized from the graphical representations of the
DSLTrans transformations and of the SyVOLT contracts, and passed to the
back-end. In the reverse direction, the proof result and counter-examples, if
any, are passed from the proof engine to the front-end.

Note that in Figure~\ref{fig:arch}, each component is annotated by a number
identifier. Following this numbering sequence, we will briefly visit each
component to describe its function and the technologies employed in its development.

SyVOLT's architectural components described in Figure~\ref{fig:arch} are
orchestrated by an ANT script from within Eclipse. This script makes sure all
components communicate and execute in the right order, and allows contract
proof to run fully automatically at the push of a button.

\begin{figure}
\centering
\includegraphics[width=0.5\textwidth]{figures/eclipse_frontend}
\caption{The SyVOLT editor within Eclipse}
\label{fig:eclipse_frontend}
\vspace{-.5cm}
\end{figure}


\subsection*{1. SyVOLT Contract Editor}

%how it was developed
The SyVolt contract editor, shown in Figure~\ref{fig:eclipse_frontend}, is
realized by a set of Eclipse Graphical Modeling Framework\footnote{\url{http://www.eclipse.org/modeling/gmp/}} (GMF) plugins, generated using
EuGENia~\cite{Kolovos2010a}.

%how it is used
The user can prescribe contracts (the gray, red and blue rectangle in the
left-hand side of Figure~\ref{fig:eclipse_frontend}) and propositional logic properties involving
contracts using the toolbox (right-hand side of
Figure~\ref{fig:eclipse_frontend}).

% what's it like internally.
Internally, the graphical editor reads and write two models: the domain model and the graphical model.
The graphical model can be seen as the realization of the concrete syntax by storing the coordinates and other visual information about the shapes that are shown in the graphical editor (Figure~\ref{fig:eclipse_frontend}). 
In contrast, the Ecore domain model contains the abstract syntax: the essence of
the elements that form the contracts and the propositional properties.
%bridge to the next section
The domain model is the artifact used when generating the models for the construction of the proof.


\begin{comment}
Eugenia consists of a set of annotations that are attached to the metamodel of
SyVOLT contracts. An annotation can, for instance, specify that an atomic
contract will be drawn as a rounded rectangle, with a specific color and a label
that is equal to the name attribute of the contract.
The annotations are not very expressive but they contain the essential
information to generate a set of GMF models that, ultimately, describe a basic
usable graphical editor.
Each generated GMF model is concerned with one aspect of the editor and can be
further customized to our needs. GMF models are much more expressive than the
Eugenia annotations.
For instance, the generated GMF tooling model prescribes the kind of tools that
will be available in the toolbox (shown in the right of
Figure~\ref{fig:eclipse_frontend}), their icons, labels, etc\ldots

From the set of GMF models, a set of eclipse plugins are synthesized.
These make up SyVolt graphical editor\footnote{A fixed structure textual editor
is also provided but this editor lacks several usability improvements and,
hence, is not appropriate to model SyVolt contracts.}.

The original DSLTrans transformation is manipulated using the DSLTrans editor
and is serialized in XMI format. Similarly, SyVOLT contracts are manipulated with the SyVolt graphical
editor and serialized in XMI format.

\end{comment}



\subsection*{2. Generating Rule and Contract Artifacts}
\label{sec:gen_models_mt}

The translation of DSLTrans transformations and SyVOLT contracts into Himesis
models and model transformation rules was achieved using the EGL. Specifically, two EGL model-to-text
transformations were used: one to generate the Himesis models
representing a DSLTrans transformation (marked (a) in Figure~\ref{fig:arch});
and another one to generate the T-Core model transformation rules necessary to prove
SyVOLT contracts (marked (b) in Figure~\ref{fig:arch}).

From the preEcore models of the DSLTrans model transformation under analysis and
of the contracts to be proved, the EGL transformations produce a number of Python
classes that inherit from the Himesis library and that represent models and
T-Core transformation rules. These classes will be subsequently loaded in memory
and manipulated by the proof engine back-end.


% Note that all these artifacts are produced as
% Python text files containing Himesis format that are needed to automatically
% perform the required proofs.
% containing models of the DSLTrans transformation being analysed
% and the T-Core model transformation rules necessary for the property proof step.

% Note that, while this approach of building models and model transformations as
% textual files is sufficient for our needs, it is not ideal from the model-driven
% development viewpoint. Building models as textual files forces us to deal to
% the accidental complexity of Python code and of the Himesis format, making the
% generation of these artifacts more difficult and error prone than it needs to
% be. The issues encountered here expose one of the current important fragilities
% of model-driven development: the difficulty of exchanging high-level data
% between different modelling frameworks~\levi{cite benoit combemale here}.


% As opposed to the XMI format which declaratively represents the graph, the
% Himesis representation of a graph is given by Python code referencing the
% Himesis library. When this code is executed it loads the Himesis graph to
% memory.

% Since Himesis is a textual format, identifiers are necessary to correctly
% describe the associations between graph nodes. The EGL transformations make sure
% these are correctly generated.


\subsection*{3. Generating Artifacts for Path Condition Generation}

Additional artifacts need to be generated for the property proof
algorithm to execute. These are the \emph{path condition generation} T-Core
model transformations (marked (c) in Figure~\ref{fig:arch}) that are needed to
perform the model manipulations for a set of path conditions. These model
transformation blocks allow combining the rules of a DSLTrans model
transformation into a set of \emph{path conditions}, as per the algorithm
described in~\cite{Lucio2014}.

This additional model transformation generation step is achieved by the PyRamify
component. PyRamify is a Python script that takes as input Himesis graphs
representing the rules in the transformation. Then, T-Core matchers and
rewriters for each rule are produced. While matchers allow the contract prover to determine
if and how rules might combine with a path condition under construction,
rewriters combine the right-hand side of a DSLTrans rule with that same path
condition.

% PyRamify is also responsible for producing additional T-Core model
% transformations used for analysing how the DSLTrans rules in a transformation
% depend on each other. Explain this further??

% Currently, PyRamify is implemented as a Python script, as we were unable to
% implement at the time automatic creation of higher-order artifacts (in our
% case, T-Core transformations) from models in another modelling tool.
% This was due to the mix of different modelling abstraction levels present in
% SyVOLT and that led to a number of technical difficulties.
% 
% We have recently learned that, alternatively, \emph{path condition generation
% transformations} could be built directly from the Ecore representation of
% DSLTrans rules by using EGL technology -- as was done for the artifacts
% described in Section~\ref{sec:gen_models_mt}. Nevertheless, a preferred solution for us would
% be to use model transformation technology that that explicitely allows
% building model transformations from simple models. To the best of our knowledge,
% model transformation technology that allows moving between models of a given
% order $n$ and an order $n+1$ is not yet available in the modelling community.

\subsection*{4. Path Condition Generation}
\label{sec:path_cond_gen}

Once the required artifacts have been produced, the prover moves onto the path
condition generation step. Path condition generation is operationally achieved
by executing the appropriate T-Core transformation rules (artifact (c) in
Figure~\ref{fig:arch}), following the ordering of the rules in the DSLTrans transformation being analysed.
The path condition generation algorithm starts from the empty path condition,
representing the case where no rules in the transformation have executed. Then, each rule in the DSLTrans model transformation under analysis
is combined with the path conditions generated thus far, using a T-Core
transformation rule matcher and rewriter. As each rule is considered, the set
of path conditions will grow to represent all allowed rules combinations. As the execution of a rule in a DSLTrans
transformation may depend on the previous execution of other rules, such
dependencies are verified by the path condition generation
transformations in order to exclude impossible rule combinations.

% The final set of path conditions produced by the algorithm will then abstract
% the infinite set of all concrete transformation executions. This is further
% described in~\cite{Lucio2014}, along with a formal discussion of the validity
% and completeness of this work.

% Note that path condition generation is a complex task, difficult to fully
% model as a single model transformation. As such, the algorithm in this component
% has been implemented as a Python script that is responsible for scheduling all the
% path condition generation transformations in the right order. This programmatic level of granularity of
% manipulation has allowed us to build many optimizations to speed up the path
% condition generation process. Examples of such optimizations are the caching of
% expensive match operations, or the parallelization of the path condition generation algorithm. 
 
\subsection*{5. Contract Proof}

The \emph{contract proof} component requires two inputs: the \emph{path
conditions set} built by the path condition generator (marked (d) in Figure~\ref{fig:arch}); the \emph{contract
proof T-Core transformations} (marked (b) in Figure~\ref{fig:arch}).
Pre-/post- condition contracts form an implication, which needs to be checked
for each path condition provided as input to the contract proof component. For a
contract to hold on a DSLTrans model transformation, the contract's implication
should hold for every path condition generated for that model transformation.
For each submodel of the path condition that is isomorphic to the
pre-condition's model, the appropriate property T-Core matcher from the artifact marked (b) will try to find
a submodel of the path condition that is isomorphic to the post-condition's
model. 
% As in for the path condition generation component described in
% Section~\ref{sec:path_cond_gen}, the contract proof algorithm has also been
% implemented as a Python script that schedules all the property proof T-Core
% transformations appropriately. As before, using fine-grained programmatic
% manipulations of the models and model transformations involved in the contract
% proof algorithm has allowed us to implement optimization strategies during
% contract proof. The main of these strategies is the skipping of path conditions
% for which the contract is sure to hold without further analysis, given
% some existing cached information about similar path conditions.



%
%\section{Mapping ATL into DSLTrans}
%\label{sec:mapping}
%\input{text/mapping}
%
%% \section{Transformation to DSLTrans}
%% \label{sec:transformation}
%% \input{text/hot}
%
%%\section{Properties and Property Prover}
%%\label{sec:prover}
%%\input{text/prover}
%
%\section{Results and Discussion}
%\label{sec:results}
%\input{text/results}
%
%
%\section{Related Work}
%\label{sec:related}
%\input{text/related}
%
%\section{Conclusion}
%\label{sec:conclusion}
%\input{text/conclusion}\vspace{-.5cm}

\section*{Acknowledgements}
The authors warmly thank Gehan Selim for her work on the
implementation of the contract prover. Bentley Oakes and Levi L\'ucio are
researchers working for the NECSIS project, funded by the Automotive Partnership
Canada. %The work of Javier Troya and Manuel Wimmer is funded by the European
%Commission under ICT Policy Support Programme, grant no. 317859.

\bibliographystyle{abbrv}
\bibliography{models2015}
\end{document}
